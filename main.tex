\documentclass[a4paper, 12pt]{article}

\usepackage[english]{babel}
\usepackage[utf8]{inputenc}
\usepackage{parskip}
\usepackage{titlesec}
\usepackage{array}
\usepackage{setspace} \doublespacing
\titleformat{\section}
  {\normalfont\fontsize{17}{15}\bfseries}{\thesection}{1em}{}
 \usepackage{epigraph}
\usepackage{graphicx}
\usepackage{float}
\usepackage{verbatim}
\usepackage{braket}
\usepackage{amsmath,amsthm,amssymb}
\usepackage{mathtext}
\usepackage[utf8]{inputenc}
\usepackage{cmap}      
\usepackage{tabularx}

\usepackage{setspace}
\usepackage{subcaption}
\usepackage{mathtext}         
\usepackage[T2A]{fontenc}
\usepackage[utf8]{inputenc}    

\usepackage{tablefootnote}
\usepackage{threeparttable}
\usepackage{threeparttablex}
\usepackage{setspace}
\newtheorem{prop}{Proposition}

\usepackage{csquotes}
\usepackage[style=apa, sorting=nty, bibencoding=utf8]{biblatex}
\addbibresource{references.bib}
\DeclareLanguageMapping{british}{british-apa}


\usepackage[dvipsnames]{xcolor}
\usepackage{hyperref}
\hypersetup{
    colorlinks=true,
    linkcolor=red,
    citecolor =red,
    filecolor=red,      
    urlcolor=red,
}

\usepackage{moreverb,savetrees}
\immediate\write18{texcount -char -sum \jobname.tex > /tmp/wordcount.tex}

\usepackage{enumerate}
\usepackage{listings}
\usepackage{hyperref}
\usepackage{fancyhdr}
\usepackage{istgame}
\usepackage{float}
\usepackage{hyperref}
\usepackage{fancyhdr}
\usepackage{tikz}
\usetikzlibrary{through}
\usepackage{pgfplots}
\usepackage[overload]{empheq} 
% Margins
\usepackage{geometry} 
    \geometry{top=8mm}
    \geometry{bottom=25mm}
    \geometry{left=20mm}
    \geometry{right=20mm}
% Colour table cells
% Get larger line spacing in table
\newcommand{\tablespace}{\\[1mm]}
\newcommand\Tstrut{\rule{0pt}{2.6ex}}         % = `top' strut
\newcommand\tstrut{\rule{0pt}{2.0ex}}         % = `top' strut
\newcommand\Bstrut{\rule[-0.9ex]{0pt}{0pt}}   % = `bottom' strut

%%%%%%%%%%%%%%%%%
%     Title     %
%%%%%%%%%%%%%%%%%
\title{
Term Paper \\  Breaking down authoritarian elections: which components affect democratization prospects}
\author{Aleksasndr Izosenkov \\ BPT 201}
\date{\today}
\setlength\parindent{24pt}
\begin{document}

\doublespacing

\thispagestyle{empty} 

\begin{center}
NATIONAL RESEARCH UNIVERSITY
\\
HIGHER SCHOOL OF ECONOMICS
\\
FACULTY OF SOCIAL SCIENCE

\end{center}
\vspace{12ex}

\begin{center}
\textbf{TERM PAPER}\\
\vspace{2ex}
\textbf{Breaking down authoritarian elections: which components affect democratization prospects}\\
\vspace{2ex}
Political Science
\end{center}

\vspace{5ex}
\begin{flushright}
\hspace{40ex}
Student\\
Izosenkov Aleksandr \\

\vspace{5ex}
Supervisor\\
Ekim Arbatli, Associate Professor,\\ National Research University \\ Higher School of Economics

\vspace{5ex}
Advisor\\
Alla Tamobvtseva, Assistant Professor, \\National Research University\\ Higher School of Economics


\end{flushright}

\begin{center}
\vfill
Moscow, 2022
\end{center}

\newpage

\tableofcontents

\newpage

\begin{abstract}
An autocratic breakdown does not imply that the regime will successfully make its path to democracy. To make a successful transit, a wide range of institutions is needed to be established. However, political scientists still doubt which institutions and how should they be established in order for democracy to consolidate successfully. In this work I examine some properties of the elections held under authoritarianism and their impact on the path of democratic consolidation. \end{abstract}

\section*{Introduction}
Current political research agenda has a substantial focus on political regimes and transitions in particular. The newest research in this area is driven by large-N statistical and economical studies. Most of the prominent studies in the area suggest a well-known pool of features of the regime which are associated with democratization. Economic development or the processes which can be associated with it (distribution made by rentier state, for example), international climate, urbanisation and education rates can be seen as factors which propel the process of democratization \parencite{geddes_what_2007}. \par  Another perspective to look from on democratization is to investigate what factors could help the autocratic incumbent to stay in power. It is widely accepted that the success of the regime depends on its ability to institutionalise in a proper way \parencite{gerschewski_three_2013}.  \par From another point of view, some analysts insist that the success of democratic consolidation is also dependent on the way which institutions are set \parencite{schedler_what_1998}. Elections are one of the core institutions of democracy even in its minimalist view \parencite{odonnell_democracy_2001}. Thus it is possible to conclude that elections could play a huge role during the process of democracy consolidation as they are crucial institution of democracy. \par However, most of the research on elections in the context of political regimes and regime transitions is focused "on the two poles". Some scholars extensively study elections in nondemocratic regimes, others investigate their properties in democracies. Surprisingly, the potential link between elections in autocratic regime and its path of democratic consolidation in case of an autocratic breakdown is not studied well. Following a well known tradition of historical institutionalism I argue that institutional context mediates the result one or another force will generate \parencite{hall_political_1996}. \par The main motivation of this paper is that we do not know a lot about the potential impact the properties of the authoritarian elections institute could have on the path of the democratic consolidation. The research question is \textbf{which of the election properties have an impact on the path of the democratization?}. The object of the research are the authoritarian regime's election properties. The subject matter of the research is the impact these properties could have on democratization. The main goal of the paper is to find out which of the elections characteristics have an impact on the path of democratic consolidation after democratic breakdown and to what extent. To achieve it, I run an OLS regression analysis and investigate the results. \par


\section*{Literature review}
This section contains a detailed overview how scholars approached the topic of elections and democratization in this context. \par
Most of the previous studies on the topic of elections in nondemocratic regimes could be divided in three groups: the studies of the role of the elections in authoritarian regime, the studies of actors' behaviour and its implications on the regime, and the studies of authoritarian elections in the context of democratization \parencite{gandhi_elections_2009}. The studies of actors' behaviour and its implications on the regime are not in the scope of this research because this group of studies takes the regime and the electoral institution for granted and on "as it is" basis and makes a focus on the actors' choices on their participation in elections, but it studies neither the institute of the elections nor the democratization process. \par
Why does a dictator need to hold the elections if they are typically seen as a democratic means of political contestation and by their nature could force a legitimate transit of power? Does an incumbent want to be ousted? A large share of the studies of the elections' role in the authoritarian regime considers them not as a threat to the incumbent, but as an instrument for them to prolong their political career. For example, elections can be seen as a means of co-optation. In a broader context of institutionalisation Jennifer Gandhi and Adam Przeworski argue that by creating legislatures and allowing parties to participate in elections an incumbent can create a special arena where it is possible to co-opt opposition and make policy concessions, thus prolonging their time in power. \parencite{gandhi_authoritarian_2007}. Another way elections could help an autocrat is through legitimation process. Barbara Geddes shows, that overwhelmingly high poll and election results could make political rivals feel that there is no base for their support.  \parencite{geddes_role_2005}. \par % Сюда еще добавить
From the perspective of authoritarian elections' role in democratization most of the studies examine how the elections by themselves and actors which participate in them could affect the regime and/or force the regime transition. For example, Mark Howard and Philip Roessler show that authoritarian elections could lead to "Liberalising election outcome" if opposition forces effectively organise themselves and build coalition or support one candidate jointly \parencite{howard_liberalizing_2006}. Beatriz Magaloni comes to the same conclusion but then adds, that not only opposition consolidation could force the autocrat to stop rigging the election, but also a massive civil protest or disobedience. \par However, the elections by themselves as were studied much less. Jason Brownlee in his book shows, that elections by themselves do not have a statistically significant effect on possible autocratic breakdown \parencite{brownlee_authoritarianism_2007}. All these studies view elections as a main factor which could directly trigger the autocratic breakdown and furtrer transition. \par
Nevertheless, some scholars argue that the elections themselves could have not direct (in case of a "liberalising election outcome"), but also indirect effect. Steffan Lindberg shows, that elections could affect factors, which influence the cost of repression and the cost of toleration. In the case of democratization process, elections increase cost of repression and decrease cost of toleration. \parencite{lindberg_democratization_2009}. \par
One of the limitations of all of these papers is that all of them consider elections as a potential direct or indirect trigger of the event of democratization or autocratization. They study only the short-term impact of the elections and and the events they could trigger. Studying elections in this way could give an opportunity to concentrate on factors which can cause one or another result based on an election, but not the election properties and their effect.
\par Amanda B. Edgel et al. address this issue in their study. They explore the impact multiparty elections could have on democratization and come to the conclusion that on global sample reiterating multiparty elections have a positive impact on democratization \parencite{edgell_when_2018}.
\par In my research I want to go further and examine the effect of election properties not on overall change in the rate of democracy, but only their impact on the path of democratic consolidation.

\section*{Hypothesis and theory}
Why do elections matter for democracy? As it was mentioned above, I argue that elections are the core democratic institution. Some scholars think that elections are the basis of democracy because only by participating in elections could an ordinary man participate in politics. This approach to democracy is often called "minimalist", but it is also possible to make a conclusion that this approach is elitistic by its nature. For example, Joseph Schumpeter suggests the procedural definition of democracy instead of the classical one and stresses that the procedural definition is effective in terms of distinguishing one government from another. It uses formal criteria and there is no need to define a hardly conceptualisable matter like common good or will of people \parencite{schumpeter_capitalism_2012}. \par
In this study I believe that the existence of the free and fair elections through which citizens could elect a representative is enough to consider a regime democratic. Nevertheless, I believe that this criteria effectively makes a classification between democracy and authoritarianism, but it lacks the possibility to describe differences between democracies by themselves. Some of them may be more advanced than others. I believe that Robert Dahl gives the definition of democracy (polyarchy) which both makes it possible to easily distinguish democracies from autocracies and describe different types of democracy. He suggests that not only the type of the elections by themselves and voter's participation in them defy the democracy, but also inclusion, control of the agenda and enlightened understanding. Some of these criteria correspond to the factors of the freedom of expression, alternative information, associational autonomy \parencite{dahl_democracy_1989}. These and other factors complement strictly procedural definition of democracy and make it possible to describe the democracy more precisely. In all, in this study I will use Dahl's concept of polyarchy to define and describe democracy. \par
What it comes to autocracy, at this point I will say that I understand it as the opposite to the democracy which I defined above. \par
Democratic consolidation is one of the crucial concepts to define and understand. As an autocratic regime faces a breakdown, a question of its future faith arises. From the transitological point of view the the success of democratization is critical factor which could describe the autocratic regime's path after its breakdown. \par
I would like to begin with the proposition that democratic consolidation is a measure of successfulness of the process of democratization. From this point of view it is beneficial to divide democracies in some groups. As David Collier and Steven Levitsky show, sometimes it could be beneficial for a researcher to use a diminished subtype, especially when there are situations when one or another feature is missing \parencite{collier_democracy_1997}. If we take the concept of polyarchy as an ideal model of democracy, then the electoral democracy could serve as one of the possible diminished subtypes, because it satisfies the procedural criteria, but fails to satisfy the criteria of broad civil and political rights. In terms of democracy consolidation, Schedler proposes a 


\section*{Methodology}


\section*{Results and discussion}

\printbibliography


\end{document}
