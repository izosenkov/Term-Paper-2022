\documentclass[a4paper, 11pt]{article}

\usepackage[english]{babel}
\usepackage[utf8]{inputenc}
\usepackage{amsmath,amssymb}
\usepackage{parskip}
\usepackage{titlesec}
\usepackage{array}
\usepackage{setspace} \doublespacing
\titleformat{\section}
  {\normalfont\fontsize{17}{15}\bfseries}{\thesection}{1em}{}
 \usepackage{epigraph}
\usepackage{graphicx}
\usepackage{float}
\usepackage{verbatim}
\usepackage{braket}
\usepackage{amsmath,amsthm,amssymb}
\usepackage{mathtext}
\usepackage[utf8]{inputenc}
\usepackage{cmap}      
\usepackage{tabularx}

\usepackage{setspace}
\usepackage{subcaption}
\usepackage{mathtext}         
\usepackage[T2A]{fontenc}
\usepackage[utf8]{inputenc}    

\usepackage{tablefootnote}
\usepackage{threeparttable}
\usepackage{threeparttablex}
\usepackage{setspace}
\newtheorem{prop}{Proposition}

\usepackage{tikz}
\usetikzlibrary{through}
\usepackage{csquotes}
\usepackage[style=apa, sorting=nty, bibencoding=utf8]{biblatex}
\addbibresource{literature.bib}
\DeclareLanguageMapping{british}{british-apa}
\usepackage[dvipsnames]{xcolor}
\usepackage{hyperref}
\hypersetup{
    colorlinks=true,
    linkcolor=red,
    citecolor =red,
    filecolor=red,      
    urlcolor=red,
}

\usepackage{moreverb,savetrees}
\immediate\write18{texcount -char -sum \jobname.tex > /tmp/wordcount.tex}

\usepackage{enumerate}
\usepackage{listings}
\usepackage{hyperref}
\usepackage{fancyhdr}
\usepackage{istgame}
\usepackage{float}
\usepackage{hyperref}
\usepackage{fancyhdr}
\usepackage{tikz}
\usetikzlibrary{through}
\usepackage{pgfplots}
\usepackage[overload]{empheq} 
% Margins
\usepackage{geometry} 
    \geometry{top=8mm}
    \geometry{bottom=25mm}
    \geometry{left=20mm}
    \geometry{right=20mm}
% Colour table cells
% Get larger line spacing in table
\newcommand{\tablespace}{\\[1mm]}
\newcommand\Tstrut{\rule{0pt}{2.6ex}}         % = `top' strut
\newcommand\tstrut{\rule{0pt}{2.0ex}}         % = `top' strut
\newcommand\Bstrut{\rule[-0.9ex]{0pt}{0pt}}   % = `bottom' strut

%%%%%%%%%%%%%%%%%
%     Title     %
%%%%%%%%%%%%%%%%%
\title{
Term Paper \\  Breaking down authoritarian elections: which components affect democratization prospects}
\author{Aleksasndr Izosenkov \\ BPT 201}
\date{\today}
\setlength\parindent{24pt}
\begin{document}

\doublespacing

\thispagestyle{empty} 

\begin{center}
NATIONAL RESEARCH UNIVERSITY
\\
HIGHER SCHOOL OF ECONOMICS
\\
FACULTY OF SOCIAL SCIENCE

\end{center}
\vspace{12ex}

\begin{center}
\textbf{TERM PAPER}\\
\vspace{2ex}
\textbf{Breaking down authoritarian elections: which components affect democratization prospects}\\
\vspace{2ex}
Political Science
\end{center}

\vspace{5ex}
\begin{flushright}
\hspace{40ex}
Student\\
Izosenkov Aleksandr \\

\vspace{5ex}
Supervisor\\
Ekim Arbatli, Associate Professor,\\ National Research University \\ Higher School of Economics

\vspace{5ex}
Advisor\\
Alla Tamobvtseva, Assistant Professor, \\National Research University\\ Higher School of Economics


\end{flushright}

\begin{center}
\vfill
Moscow, 2022
\end{center}

\newpage

\begin{abstract}
An autocratic breakdown does not imply that the regime will successfully make its path to democracy. To make a successful transit, a wide range of institutions is needed to be established. However, political scientists still doubt (and will do so in the near future) which institutions and how should they be established in order for democracy to consolidate successfully. In this work I examine some properties of the elections held under authoritarianism and their impact on the path of democratic consolidation. \end{abstract}

\section*{Introduction}
Current political research agenda has a substantial focus on political regimes and transitions in particular. Most of the prominent studies in this research suggest a well-known pool of features of the regime which are associated with democratization. Economic development, 
\section*{Literature review}
\section*{Hypothesis and theory}
\section*{Methodology}
\section*{Results and discussion}


%%%%%%%%%%%


\begin{table}[!htbp] \centering 
  \caption{} 
  \label{} 
\begin{tabular}{@{\extracolsep{5pt}}lcc} 
\\[-1.8ex]\hline 
\hline \\[-1.8ex] 
 & \multicolumn{2}{c}{\textit{Dependent variable:}} \\ 
\cline{2-3} 
\\[-1.8ex] & \multicolumn{2}{c}{mpg} \\ 
\\[-1.8ex] & (1) & (2)\\ 
\hline \\[-1.8ex] 
 hp & $-$0.068$^{***}$ &  \\ 
  & (0.010) &  \\ 
  & & \\ 
 disp &  & $-$0.041$^{***}$ \\ 
  &  & (0.005) \\ 
  & & \\ 
 Constant & 30.099$^{***}$ & 29.600$^{***}$ \\ 
  & (1.634) & (1.230) \\ 
  & & \\ 
\hline \\[-1.8ex] 
Observations & 32 & 32 \\ 
R$^{2}$ & 0.602 & 0.718 \\ 
Adjusted R$^{2}$ & 0.589 & 0.709 \\ 
Residual Std. Error (df = 30) & 3.863 & 3.251 \\ 
F Statistic (df = 1; 30) & 45.460$^{***}$ & 76.513$^{***}$ \\ 
\hline 
\hline \\[-1.8ex] 
\textit{Note:}  & \multicolumn{2}{r}{$^{*}$p$<$0.1; $^{**}$p$<$0.05; $^{***}$p$<$0.01} \\ 
\end{tabular} 
\end{table} 

\end{document}
